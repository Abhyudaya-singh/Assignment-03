\documentclass[12pt]{article}
\usepackage[english]{babel}
\usepackage{natbib}
\usepackage{url}
\usepackage[utf8x]{inputenc}
\usepackage{amsmath}
\usepackage{graphicx}
\graphicspath{{images/}}
\usepackage{parskip}
\usepackage{fancyhdr}
\usepackage{vmargin}
\setmarginsrb{3 cm}{2.5 cm}{3 cm}{2.5 cm}{1 cm}{1.5 cm}{1 cm}{1.5 cm}

\title{Future Of Healthcare}					
\author{21111004}								
\date{28 JAN 2022}								
\makeatletter
\let\thetitle\@title
\let\theauthor\@author
\let\thedate\@date
\makeatother

\pagestyle{fancy}
\fancyhf{}
\rhead{\theauthor}
\lhead{\thetitle}
\cfoot{\thepage}

\begin{document}
\begin{titlepage}
	\centering
    \includegraphics[scale = 0.20]{logo.jpg}\\[1.0 cm]	
    \textsc{\LARGE National Institute Of Technology \newline\\\\ RAIPUR}\\[2.0 CM]
    
	\textsc{\Large ASSIGNMENT 03}\\[0.5 cm]				% Course Code
	\rule{\linewidth}{0.4 mm} \\[0.4 cm]
	{ \huge \bfseries \thetitle}\\
	\rule{\linewidth}{0.4 mm} \\[1.5 cm]
	
	\begin{minipage}{0.6\textwidth}
		\begin{flushleft} \large
			\emph{Submitted To:}\\
			Saurabh Gupta\\
            Department Of Basic Biomedical Engineering\\
			\end{flushleft}
			\end{minipage}~
			\begin{minipage}{0.4\textwidth}
            
			\begin{flushright} \large
			\emph{Submitted By :}\\
			Abhyudaya Kumar Singh\\
            21111004\\
		\end{flushright}
        
	\end{minipage}\\[2 cm]
\end{titlepage}

\tableofcontents
\pagebreak

\section{Future of Healthcare}
Like many other sectors, healthcare is about to enter a period of rapid change. Longevity and the advance of new technologies and discoveries as well as innovative combinations of existing ones are among the many factors propelling patient empowerment, which is fundamentally changing how we prevent, diagnose and cure diseases. The consequences will be far-reaching, Quality healthcare will become more accessible, as will become cheaper, more efficient and more convenient.

\subsection{Future Hospitals}
\textbf{\emph{Hospitals}} are profoundly complex buildings, comprising of a wide range of services and units, from emergency rooms and operating theatres, to clinical laboratories and imaging centres to food services and housekeeping. \newline
Tomorrow's hospitals will no doubt rely more heavily on robotics and digital technologies. Many of the physical and mental tasks that doctors perform today will be automated via hardware, software, and combinations of both. That will leave hospitals with more space in addition to the space already being freed up through telemedicine and remote healthcare. which reduce the need for patient visits.\newline
\textbf{\emph{Telemedicine}} is one of the fastest-moving areas of healthcare innovation. It is not too far-fetched to imagine the hospital of the future to be a rather sparse one. mostly a place for intensive care and robot-delivered surgeries, In 25 years, expect most healthcare delivery to be done virtually rather than in person.
 Artificially intelligent smart assistants-next generation Siris and Alexas that can attend to basic everyday whims of patients will help with taking measurements and performing diagnostics. Robot carers will supervise and assist the elderly.\newline

Innovation trends in healthcare point towards a future where our health is monitored and provided continuously, wherever we are, with less and less need for bulky physical infrastructure. \emph{The hospital of the future may well be the home.}
\subsection{AI in Healthcare}
With the advent of deep-learning algorithms in the current Al renaissance, that machines have started to equal or at times exceed humans in a wide range of perceptual tasks. This has truly transformative potential in medicine.\newline
\textbf{Radiology} and \textbf{pathology} are two examples of medical specialisations that are at their core about spotting patterns and taking readings. This is significant as these specialities alone make up as much as a quarter of a typical health budget. Soon, the world's best diagnostician in most medical specialities from not just radiology and pathology, but on to oncology. dermatology, ophthalmology - will be an algorithm.\newline
 As with all automation, machines don't displace whole jobs, only aspects of them. For radiologists, for instance, their work will be more high level: we will require fewer of them, but with greater expertise.\nelwine

Doctors will in the future continue to play a vital role both in reviewing the diagnosis, helping patients to comply with their treatment course, and forming the interpersonal connections that are most vital for effective health.
\subsection{NanoTechnology in Healthcare}
While we may not be able to shrink human doctors down to microscopic size today, medicine on an atomic and molecular scale is fast becoming reality.\newline
Engineering at this size requires manipulating individual atoms.Materials at the nanoscale - in the tens of nanometers - are almost incomprehensibly tiny, making  it exceptionally challenging to work with. But gaining the ability to do so-to build molecule-sized machines that can build and manipulate their proteins or DNA-would be the most radical transformation of healthcare in centuries.\newline
Ever-smaller wearable devices that can monitor our vital signs are becoming common, but at a nanoscale we could implant them into our bodies. Nano devices could capture incredibly detailed data from deep within us, enabling doctors to personalise treatment.\newline

When they reduce further to the nanoscale, such technology would radically improve medical imaging by delivering molecular resolution. Micro machines could identify and destroy cancer cells, keeping healthy cells untouched. They could also deliver dopamine directly to the brainstem to help treat sufferers of Parkinson's. With nanorobots, we could enter the body and even redesign the genome.\newline

We can even imagine autonomous nanorobots that eventually swim deep inside us, detecting and reacting to problems as they arise. Looking back on Fantastic Voyage, it seems reality could one day exceed even the most imaginative science fiction.
\subsection{IOT in Healthcare}
One of the most hyped technological trends of the past ten years has been the promise of internet-connected everyday devices to transform everyday life by moving the internet off of our screens and into the physical environment.\newline
The ability to put cheap sensors into any object, and connect them to the internet, offers a host of benefits, particularly in healthcare. Patients save time by being monitored remotely. Those with mobility issues will also benefit. as will the hospitals and clinics, which will face lesser burdens on capacity. Doctors will also receive continuous patient data, giving them a more detailed picture of their patients' health.\newline
 Innovators are developing sensors that are woven directly into clothing, or applied to the skin like a tattoo. With current remote sensor systems, caregivers are able to set up alert thresholds such that, within a few minutes of a patient recording a metric that is out of the ordinary, doctors can conduct a tele visit or dispatch a home nurse.
While this monitoring capability is impressive, the promise of the data captured, both individually and collectively, is what will truly transform healthcare in the next ten years. With such fine-grained data gathered continuously, once imperceptible changes could, in the future, be discovered much earlier.
\end{document}
